\documentclass[]{article}
\usepackage{lmodern}
\usepackage{amssymb,amsmath}
\usepackage{ifxetex,ifluatex}
\usepackage{fixltx2e} % provides \textsubscript
\ifnum 0\ifxetex 1\fi\ifluatex 1\fi=0 % if pdftex
  \usepackage[T1]{fontenc}
  \usepackage[utf8]{inputenc}
\else % if luatex or xelatex
  \ifxetex
    \usepackage{mathspec}
  \else
    \usepackage{fontspec}
  \fi
  \defaultfontfeatures{Ligatures=TeX,Scale=MatchLowercase}
\fi
% use upquote if available, for straight quotes in verbatim environments
\IfFileExists{upquote.sty}{\usepackage{upquote}}{}
% use microtype if available
\IfFileExists{microtype.sty}{%
\usepackage{microtype}
\UseMicrotypeSet[protrusion]{basicmath} % disable protrusion for tt fonts
}{}
\usepackage[margin=1in]{geometry}
\usepackage{hyperref}
\hypersetup{unicode=true,
            pdftitle={732A90 Computational Statistics - Lab 1},
            pdfauthor={Joris van Doorn - jorva845 \textbar{} Bayu Brahmantio - baybr878 \textbar{} Ismail Khalil - ismkh208},
            pdfborder={0 0 0},
            breaklinks=true}
\urlstyle{same}  % don't use monospace font for urls
\usepackage{color}
\usepackage{fancyvrb}
\newcommand{\VerbBar}{|}
\newcommand{\VERB}{\Verb[commandchars=\\\{\}]}
\DefineVerbatimEnvironment{Highlighting}{Verbatim}{commandchars=\\\{\}}
% Add ',fontsize=\small' for more characters per line
\usepackage{framed}
\definecolor{shadecolor}{RGB}{248,248,248}
\newenvironment{Shaded}{\begin{snugshade}}{\end{snugshade}}
\newcommand{\AlertTok}[1]{\textcolor[rgb]{0.94,0.16,0.16}{#1}}
\newcommand{\AnnotationTok}[1]{\textcolor[rgb]{0.56,0.35,0.01}{\textbf{\textit{#1}}}}
\newcommand{\AttributeTok}[1]{\textcolor[rgb]{0.77,0.63,0.00}{#1}}
\newcommand{\BaseNTok}[1]{\textcolor[rgb]{0.00,0.00,0.81}{#1}}
\newcommand{\BuiltInTok}[1]{#1}
\newcommand{\CharTok}[1]{\textcolor[rgb]{0.31,0.60,0.02}{#1}}
\newcommand{\CommentTok}[1]{\textcolor[rgb]{0.56,0.35,0.01}{\textit{#1}}}
\newcommand{\CommentVarTok}[1]{\textcolor[rgb]{0.56,0.35,0.01}{\textbf{\textit{#1}}}}
\newcommand{\ConstantTok}[1]{\textcolor[rgb]{0.00,0.00,0.00}{#1}}
\newcommand{\ControlFlowTok}[1]{\textcolor[rgb]{0.13,0.29,0.53}{\textbf{#1}}}
\newcommand{\DataTypeTok}[1]{\textcolor[rgb]{0.13,0.29,0.53}{#1}}
\newcommand{\DecValTok}[1]{\textcolor[rgb]{0.00,0.00,0.81}{#1}}
\newcommand{\DocumentationTok}[1]{\textcolor[rgb]{0.56,0.35,0.01}{\textbf{\textit{#1}}}}
\newcommand{\ErrorTok}[1]{\textcolor[rgb]{0.64,0.00,0.00}{\textbf{#1}}}
\newcommand{\ExtensionTok}[1]{#1}
\newcommand{\FloatTok}[1]{\textcolor[rgb]{0.00,0.00,0.81}{#1}}
\newcommand{\FunctionTok}[1]{\textcolor[rgb]{0.00,0.00,0.00}{#1}}
\newcommand{\ImportTok}[1]{#1}
\newcommand{\InformationTok}[1]{\textcolor[rgb]{0.56,0.35,0.01}{\textbf{\textit{#1}}}}
\newcommand{\KeywordTok}[1]{\textcolor[rgb]{0.13,0.29,0.53}{\textbf{#1}}}
\newcommand{\NormalTok}[1]{#1}
\newcommand{\OperatorTok}[1]{\textcolor[rgb]{0.81,0.36,0.00}{\textbf{#1}}}
\newcommand{\OtherTok}[1]{\textcolor[rgb]{0.56,0.35,0.01}{#1}}
\newcommand{\PreprocessorTok}[1]{\textcolor[rgb]{0.56,0.35,0.01}{\textit{#1}}}
\newcommand{\RegionMarkerTok}[1]{#1}
\newcommand{\SpecialCharTok}[1]{\textcolor[rgb]{0.00,0.00,0.00}{#1}}
\newcommand{\SpecialStringTok}[1]{\textcolor[rgb]{0.31,0.60,0.02}{#1}}
\newcommand{\StringTok}[1]{\textcolor[rgb]{0.31,0.60,0.02}{#1}}
\newcommand{\VariableTok}[1]{\textcolor[rgb]{0.00,0.00,0.00}{#1}}
\newcommand{\VerbatimStringTok}[1]{\textcolor[rgb]{0.31,0.60,0.02}{#1}}
\newcommand{\WarningTok}[1]{\textcolor[rgb]{0.56,0.35,0.01}{\textbf{\textit{#1}}}}
\usepackage{graphicx,grffile}
\makeatletter
\def\maxwidth{\ifdim\Gin@nat@width>\linewidth\linewidth\else\Gin@nat@width\fi}
\def\maxheight{\ifdim\Gin@nat@height>\textheight\textheight\else\Gin@nat@height\fi}
\makeatother
% Scale images if necessary, so that they will not overflow the page
% margins by default, and it is still possible to overwrite the defaults
% using explicit options in \includegraphics[width, height, ...]{}
\setkeys{Gin}{width=\maxwidth,height=\maxheight,keepaspectratio}
\IfFileExists{parskip.sty}{%
\usepackage{parskip}
}{% else
\setlength{\parindent}{0pt}
\setlength{\parskip}{6pt plus 2pt minus 1pt}
}
\setlength{\emergencystretch}{3em}  % prevent overfull lines
\providecommand{\tightlist}{%
  \setlength{\itemsep}{0pt}\setlength{\parskip}{0pt}}
\setcounter{secnumdepth}{0}
% Redefines (sub)paragraphs to behave more like sections
\ifx\paragraph\undefined\else
\let\oldparagraph\paragraph
\renewcommand{\paragraph}[1]{\oldparagraph{#1}\mbox{}}
\fi
\ifx\subparagraph\undefined\else
\let\oldsubparagraph\subparagraph
\renewcommand{\subparagraph}[1]{\oldsubparagraph{#1}\mbox{}}
\fi

%%% Use protect on footnotes to avoid problems with footnotes in titles
\let\rmarkdownfootnote\footnote%
\def\footnote{\protect\rmarkdownfootnote}

%%% Change title format to be more compact
\usepackage{titling}

% Create subtitle command for use in maketitle
\providecommand{\subtitle}[1]{
  \posttitle{
    \begin{center}\large#1\end{center}
    }
}

\setlength{\droptitle}{-2em}

  \title{732A90 Computational Statistics - Lab 1}
    \pretitle{\vspace{\droptitle}\centering\huge}
  \posttitle{\par}
    \author{Joris van Doorn - jorva845 \textbar{} Bayu Brahmantio - baybr878
\textbar{} Ismail Khalil - ismkh208}
    \preauthor{\centering\large\emph}
  \postauthor{\par}
      \predate{\centering\large\emph}
  \postdate{\par}
    \date{29 January 2020}


\begin{document}
\maketitle

\#Q1: Be careful when comparing

\emph{Consider the following two R code snippets}

\begin{Shaded}
\begin{Highlighting}[]
\NormalTok{x1 <-}\StringTok{ }\DecValTok{1}\OperatorTok{/}\DecValTok{3}\NormalTok{; x2 <-}\StringTok{ }\DecValTok{1}\OperatorTok{/}\DecValTok{4}
\ControlFlowTok{if}\NormalTok{(x1}\OperatorTok{-}\NormalTok{x2}\OperatorTok{==}\DecValTok{1}\OperatorTok{/}\DecValTok{12}\NormalTok{)\{}
  \KeywordTok{print}\NormalTok{(}\StringTok{"Subtraction is correct"}\NormalTok{)}
\NormalTok{\}}\ControlFlowTok{else}\NormalTok{\{}
  \KeywordTok{print}\NormalTok{(}\StringTok{"Subtraction is wrong"}\NormalTok{)}
\NormalTok{\}}
\end{Highlighting}
\end{Shaded}

\begin{verbatim}
## [1] "Subtraction is wrong"
\end{verbatim}

\begin{Shaded}
\begin{Highlighting}[]
\NormalTok{x3 <-}\StringTok{ }\DecValTok{1}\NormalTok{; x4 <-}\StringTok{ }\DecValTok{1}\OperatorTok{/}\DecValTok{2}
\ControlFlowTok{if}\NormalTok{(x3}\OperatorTok{-}\NormalTok{x4}\OperatorTok{==}\DecValTok{1}\OperatorTok{/}\DecValTok{2}\NormalTok{)\{}
  \KeywordTok{print}\NormalTok{(}\StringTok{"Subtraction is correct"}\NormalTok{)}
\NormalTok{\}}\ControlFlowTok{else}\NormalTok{\{}
  \KeywordTok{print}\NormalTok{(}\StringTok{"Subtraction is wrong"}\NormalTok{)}
\NormalTok{\}}
\end{Highlighting}
\end{Shaded}

\begin{verbatim}
## [1] "Subtraction is correct"
\end{verbatim}

\#\#1.

\emph{Check the results of the snippets. Comment what is going on.}

So if we would do these calculations manually we would expect both
calculations to return the correct answer. Yet the first one fails. I
suspect that it has to do with the fact that 1/3 has a repeating
infinite decimals and R creates rounding errors when perfroming these
calculations. Let's explore:

\begin{Shaded}
\begin{Highlighting}[]
\KeywordTok{sprintf}\NormalTok{(}\StringTok{"%.20f"}\NormalTok{, (x1}\OperatorTok{-}\NormalTok{x2))}
\end{Highlighting}
\end{Shaded}

\begin{verbatim}
## [1] "0.08333333333333331483"
\end{verbatim}

\begin{Shaded}
\begin{Highlighting}[]
\KeywordTok{sprintf}\NormalTok{(}\StringTok{"%.20f"}\NormalTok{, }\DecValTok{1}\OperatorTok{/}\DecValTok{12}\NormalTok{)}
\end{Highlighting}
\end{Shaded}

\begin{verbatim}
## [1] "0.08333333333333332871"
\end{verbatim}

As becomes clear the two values in the first computation do indeed not
equal to each other.

\#\#2.

\emph{If there are any problems, suggest improvements.}

The problem lies in rounding errors, so this might be solved by using
less decimal points. Let's try.

\begin{Shaded}
\begin{Highlighting}[]
\NormalTok{x1 <-}\StringTok{ }\DecValTok{1}\OperatorTok{/}\DecValTok{3}\NormalTok{; x2 <-}\StringTok{ }\DecValTok{1}\OperatorTok{/}\DecValTok{4}
\ControlFlowTok{if}\NormalTok{(}\KeywordTok{round}\NormalTok{((x1}\OperatorTok{-}\NormalTok{x2), }\DataTypeTok{digits =} \DecValTok{15}\NormalTok{) }\OperatorTok{==}\StringTok{ }\KeywordTok{round}\NormalTok{((}\DecValTok{1}\OperatorTok{/}\DecValTok{12}\NormalTok{), }\DataTypeTok{digits =} \DecValTok{15}\NormalTok{))\{}
  \KeywordTok{print}\NormalTok{(}\StringTok{"Subtraction is correct"}\NormalTok{)}
\NormalTok{\}}\ControlFlowTok{else}\NormalTok{\{}
  \KeywordTok{print}\NormalTok{(}\StringTok{"Subtraction is wrong"}\NormalTok{)}
\NormalTok{\}}
\end{Highlighting}
\end{Shaded}

\begin{verbatim}
## [1] "Subtraction is correct"
\end{verbatim}

This seems to work.

\newpage

\#Q2: Derivative

\emph{From the defintion of a derivative a popular way of computing it
at a point x is to use a small \(\epsilon\) and the formula}

\[f'(x) = \frac{f(x + \epsilon) - f(x)}{\epsilon}\]

\#\#1.

\emph{Write your own R function to calculate the derivative of f(x) = x
in this way with \(\epsilon = 10^{15}\)}���15.*

\begin{Shaded}
\begin{Highlighting}[]
\NormalTok{f <-}\StringTok{ }\ControlFlowTok{function}\NormalTok{(x)\{}
  \KeywordTok{return}\NormalTok{(x)}
\NormalTok{\}}

\NormalTok{myderiv <-}\StringTok{ }\ControlFlowTok{function}\NormalTok{(f, x, e)\{}
\NormalTok{  deriv <-}\StringTok{ }\NormalTok{(}\KeywordTok{f}\NormalTok{(x }\OperatorTok{+}\StringTok{ }\NormalTok{e) }\OperatorTok{-}\StringTok{ }\KeywordTok{f}\NormalTok{(x))}\OperatorTok{/}\NormalTok{(e)}
  \KeywordTok{return}\NormalTok{(deriv)}
\NormalTok{\}}
\end{Highlighting}
\end{Shaded}

\#\#2.

\emph{Evaluate your derivative function at x = 1 and x = 100000.}

\begin{Shaded}
\begin{Highlighting}[]
\KeywordTok{myderiv}\NormalTok{(}\DataTypeTok{f =}\NormalTok{ f, }\DataTypeTok{x =} \DecValTok{1}\NormalTok{, }\DataTypeTok{e =} \DecValTok{10}\OperatorTok{^-}\DecValTok{15}\NormalTok{)}
\end{Highlighting}
\end{Shaded}

\begin{verbatim}
## [1] 1.110223
\end{verbatim}

\begin{Shaded}
\begin{Highlighting}[]
\KeywordTok{myderiv}\NormalTok{(}\DataTypeTok{f =}\NormalTok{ f, }\DataTypeTok{x =} \DecValTok{100000}\NormalTok{, }\DataTypeTok{e =} \DecValTok{10}\OperatorTok{^-}\DecValTok{15}\NormalTok{)}
\end{Highlighting}
\end{Shaded}

\begin{verbatim}
## [1] 0
\end{verbatim}

\#\#3.

\emph{What values did you obtain? What are the true values? Explain the
reasons behind the discovered differences.}

The values we obtained are 1.110223 and 0, which are both not the true
value. The true value should be 1 for both, because ((x+e)-x)/e should
simplify to e/e, which is 1. Again, the errors are due to rounding
issues. If x is small, the small value of e has some impact on the
computation resulting in a value close to 1, but when x gets bigger the
impact of e gets smaller and is eventually neglected by R.

\newpage

\#Q3: Variance

\emph{A known formula for estimating the variance based on a vector of n
observations is}

\[Var(\overrightarrow{x}) = \frac{1}{n - 1}(\sum_{i=1}^{n}x_i^2-\frac{1}{n}(\sum_{i=1}^{n}x_i)^2)\]

\#\#1.

\emph{Write your own R function, myvar, to estimate the variance in this
way.}

\begin{Shaded}
\begin{Highlighting}[]
\NormalTok{myvar <-}\StringTok{ }\ControlFlowTok{function}\NormalTok{(x)\{}
\NormalTok{  n <-}\StringTok{ }\KeywordTok{length}\NormalTok{(x)}
\NormalTok{  variance <-}\StringTok{ }\NormalTok{(}\DecValTok{1}\OperatorTok{/}\NormalTok{(n}\DecValTok{-1}\NormalTok{)) }\OperatorTok{*}\StringTok{ }\NormalTok{(}\KeywordTok{sum}\NormalTok{(x}\OperatorTok{^}\DecValTok{2}\NormalTok{) }\OperatorTok{-}\StringTok{ }\NormalTok{((}\DecValTok{1}\OperatorTok{/}\NormalTok{n)}\OperatorTok{*}\NormalTok{(}\KeywordTok{sum}\NormalTok{(x))}\OperatorTok{^}\DecValTok{2}\NormalTok{))}
  \KeywordTok{return}\NormalTok{(variance)}
\NormalTok{\}}
\end{Highlighting}
\end{Shaded}

\#\#2.

\emph{Generate a vector x = (x1, \ldots{}, x10000) with 10000 random
numbers with mean 10\^{}8 and variance 1.}

\begin{Shaded}
\begin{Highlighting}[]
\KeywordTok{set.seed}\NormalTok{(}\DecValTok{12345}\NormalTok{, }\DataTypeTok{kind =} \StringTok{"Mersenne-Twister"}\NormalTok{, }\DataTypeTok{normal.kind =} \StringTok{"Inversion"}\NormalTok{)}
\NormalTok{data <-}\StringTok{ }\KeywordTok{rnorm}\NormalTok{(}\DecValTok{10000}\NormalTok{, }\DataTypeTok{mean=}\DecValTok{10}\OperatorTok{^}\DecValTok{8}\NormalTok{, }\DataTypeTok{sd=}\DecValTok{1}\NormalTok{)}
\end{Highlighting}
\end{Shaded}

\#\#3.

\emph{For each subset Xi = \{x1, \ldots{}, xi\}, i = 1, \ldots{}, 10000
compute the difference Yi = myvar(Xi)-var(Xi), where var(Xi) is the
standard variance estimation function in R. Plot the dependence Yi on i.
Draw conclusions from this plot. How well does your function work? Can
you explain the behaviour?}

\begin{Shaded}
\begin{Highlighting}[]
\NormalTok{Yi <-}\StringTok{ }\DecValTok{0}

\ControlFlowTok{for}\NormalTok{(i }\ControlFlowTok{in} \DecValTok{1}\OperatorTok{:}\KeywordTok{length}\NormalTok{(data))\{}
\NormalTok{  Yi[i] <-}\StringTok{ }\KeywordTok{myvar}\NormalTok{(data[}\DecValTok{1}\OperatorTok{:}\NormalTok{i]) }\OperatorTok{-}\StringTok{ }\KeywordTok{var}\NormalTok{(data[}\DecValTok{1}\OperatorTok{:}\NormalTok{i])}
\NormalTok{\}}

\NormalTok{i <-}\StringTok{ }\KeywordTok{c}\NormalTok{(}\DecValTok{1}\OperatorTok{:}\DecValTok{10000}\NormalTok{)}
  
\KeywordTok{plot}\NormalTok{(i, Yi)}
\end{Highlighting}
\end{Shaded}

\includegraphics{732A90_Lab1_Group02_files/figure-latex/unnamed-chunk-9-1.pdf}

As you can see in the graph this variance implementation is quite far
off from the actual variance, because otherwise all the points would be
close to 0. We suspect that the difference is caused by the number of
summations we use and in the way R treats the numbers and how many
decimals get included by default.

\#\#4.

\emph{How can you better implement a variance estimator? Find and
implement a formula that will give the same results as var()?}

We want to attempt the following:

\[Var(X) = \frac{\sum_{i=1}^{n}(x_i - \overline{x})^2}{n-1}\]

\begin{Shaded}
\begin{Highlighting}[]
\NormalTok{newvar <-}\StringTok{ }\ControlFlowTok{function}\NormalTok{(x)\{}
\NormalTok{  n <-}\StringTok{ }\KeywordTok{length}\NormalTok{(x)}
\NormalTok{  xbar <-}\StringTok{ }\KeywordTok{rep}\NormalTok{(}\KeywordTok{mean}\NormalTok{(x), n)}
\NormalTok{  variance <-}\StringTok{ }\KeywordTok{sum}\NormalTok{(((x }\OperatorTok{-}\StringTok{ }\NormalTok{xbar)}\OperatorTok{^}\DecValTok{2}\NormalTok{))}\OperatorTok{/}\NormalTok{(n}\DecValTok{-1}\NormalTok{)}
  \KeywordTok{return}\NormalTok{(variance)}
\NormalTok{\}}

\NormalTok{Yi2 <-}\StringTok{ }\DecValTok{0}

\ControlFlowTok{for}\NormalTok{(i }\ControlFlowTok{in} \DecValTok{1}\OperatorTok{:}\KeywordTok{length}\NormalTok{(data))\{}
\NormalTok{  Yi2[i] <-}\StringTok{ }\KeywordTok{newvar}\NormalTok{(data[}\DecValTok{1}\OperatorTok{:}\NormalTok{i]) }\OperatorTok{-}\StringTok{ }\KeywordTok{var}\NormalTok{(data[}\DecValTok{1}\OperatorTok{:}\NormalTok{i])}
\NormalTok{\}}

\NormalTok{i <-}\StringTok{ }\KeywordTok{c}\NormalTok{(}\DecValTok{1}\OperatorTok{:}\DecValTok{10000}\NormalTok{)}

\KeywordTok{plot}\NormalTok{(i, Yi2)}
\end{Highlighting}
\end{Shaded}

\includegraphics{732A90_Lab1_Group02_files/figure-latex/unnamed-chunk-10-1.pdf}

This seems to work better. As you can see in the graph the difference in
variance converges towards 0, meaning no difference between our new
variance implementation and R's variance method.

\newpage

\#Q4: Linear Algebra

\emph{The Excel file ``tecator.xls'' contains the results of a study
aimed to investigate whether a near-infrared absorbance spectrum and the
levels of moisture and fat can be used to predict the protein content of
samples of meat. For each meat sample the data consists of a 100 channel
spectrum of absorbance records and the levels of moisture (water), fat
and protein. The absorbance is -log10 of the transmittance measured by
the spectrometer. The moisture, fat and protein are determined by
analytic chemistry. The worksheet you need to use is ``data'' (or file
``tecator.csv''). It contains data from 215 samples of finely chopped
meat. The aim is to fit a linear regression model that could predict
protein content as function of all other variables.}

\#\#1.

\emph{Import the data set to R}

\begin{Shaded}
\begin{Highlighting}[]
\NormalTok{data <-}\StringTok{ }\KeywordTok{read_excel}\NormalTok{(}\StringTok{"tecator.xls"}\NormalTok{)}
\NormalTok{y_index <-}\StringTok{ }\KeywordTok{grep}\NormalTok{(}\StringTok{"Protein"}\NormalTok{, }\KeywordTok{colnames}\NormalTok{(data))}
\NormalTok{y <-}\StringTok{ }\KeywordTok{as.matrix}\NormalTok{(data}\OperatorTok{$}\NormalTok{Protein)}
\NormalTok{X <-}\StringTok{ }\KeywordTok{as.matrix}\NormalTok{(data[,}\OperatorTok{-}\NormalTok{y_index])}
\end{Highlighting}
\end{Shaded}

\#\#2.

\emph{Optimal regression coeffcients can be found by solving a system of
type \(A\overline{\beta} = \overline{b}\) where \(A=X^TX\) and
\(\overline{b} = X^T\overline{y}\). Compute A and \(\overline{b}\) for
the given data set. The matrix X are the observations of the absorbance
records, levels of moisture and fat, while \(\overline{y}\) are the
protein levels}

\begin{Shaded}
\begin{Highlighting}[]
\NormalTok{A <-}\StringTok{ }\KeywordTok{t}\NormalTok{(X) }\OperatorTok\StringTok{ }\NormalTok{X}
\NormalTok{b <-}\StringTok{  }\KeywordTok{t}\NormalTok{(X) }\OperatorTok\StringTok{ }\NormalTok{y  }
\end{Highlighting}
\end{Shaded}

\#\#3.

\emph{Try to solve Abeta = b with default solver solve(). What kind of
result did you get? How can this result be explained?}

\begin{Shaded}
\begin{Highlighting}[]
\KeywordTok{solve}\NormalTok{(A, b)}
\end{Highlighting}
\end{Shaded}

The error that we get is: ``Error in solve.default(A, b): system is
computationally singular: reciprocal condition number = 3.02468e-17''.
This means that the matrix \(A\) is not invertible.

\#\#4.

\emph{Check the condition number of the matrix A (function kappa()) and
consider how it is related to your conclusion in step 3.}

\begin{Shaded}
\begin{Highlighting}[]
\KeywordTok{kappa}\NormalTok{(A)}
\end{Highlighting}
\end{Shaded}

\begin{verbatim}
## [1] 4.274694e+15
\end{verbatim}

The kappa of A is quite large, which explains why solve() does not work.
Kappa in this context is the condition number, which is supposed to
measure the sensitivity of the response of the output to changes in the
input. If this kappa is large, it means that small changes in the input
values have a large impact on the output values.\\
If the value of kappa is very large compared to 1, the matrix is
ill-conditioned which implies that its inverse cannot be calculated
accurately. Since \(kappa(A)\) is a large value, \(A\) is an
ill-conditioned matrix which results in singularity when computing its
inverse.

\#\#5.

\emph{Scale the data set and repeat steps 2-4. How has the result
changed and why?}

\begin{Shaded}
\begin{Highlighting}[]
\NormalTok{data2 <-}\StringTok{ }\KeywordTok{scale}\NormalTok{(data)}
\NormalTok{y_index <-}\StringTok{ }\KeywordTok{grep}\NormalTok{(}\StringTok{"Protein"}\NormalTok{, }\KeywordTok{colnames}\NormalTok{(data2))}
\NormalTok{y <-}\StringTok{ }\KeywordTok{as.matrix}\NormalTok{(data2[,y_index])}
\NormalTok{X <-}\StringTok{ }\KeywordTok{as.matrix}\NormalTok{(data2[,}\OperatorTok{-}\NormalTok{y_index])}
\NormalTok{A <-}\StringTok{ }\KeywordTok{t}\NormalTok{(X) }\OperatorTok\StringTok{ }\NormalTok{X}
\NormalTok{b <-}\StringTok{  }\KeywordTok{t}\NormalTok{(X) }\OperatorTok\StringTok{ }\NormalTok{y  }
\KeywordTok{solve}\NormalTok{(A, b)}
\end{Highlighting}
\end{Shaded}

\begin{verbatim}
##                     [,1]
## Sample        -0.0120159
## Channel1    -102.3669689
## Channel2    -248.5383275
## Channel3     412.7916359
## Channel4    -180.8159605
## Channel5     488.0241928
## Channel6    -117.5282387
## Channel7    -191.7399928
## Channel8      78.8198264
## Channel9    -137.2970747
## Channel10    219.9621095
## Channel11   -308.2173340
## Channel12   -317.2517395
## Channel13    680.5116039
## Channel14   -376.6516565
## Channel15    131.0865458
## Channel16   -298.1785020
## Channel17    168.6764043
## Channel18    278.1307690
## Channel19   -291.8090262
## Channel20    -30.6267135
## Channel21    571.2811368
## Channel22  -1412.2607600
## Channel23   1986.4880884
## Channel24  -1569.3869206
## Channel25    711.3571490
## Channel26   -110.0583088
## Channel27    -10.5583156
## Channel28   -206.4807821
## Channel29    447.1327293
## Channel30   -653.2555738
## Channel31    495.5693926
## Channel32    302.0341838
## Channel33   -321.7226564
## Channel34   -357.7911331
## Channel35    462.4522471
## Channel36    -47.4888697
## Channel37   -335.2394608
## Channel38    267.5643640
## Channel39   -463.9890027
## Channel40    925.9917924
## Channel41   -997.4952577
## Channel42    423.6208334
## Channel43    432.0899296
## Channel44   -548.5199342
## Channel45   -120.3591597
## Channel46    577.3962163
## Channel47   -323.1874931
## Channel48    -56.8769166
## Channel49    -71.7587985
## Channel50    372.1231270
## Channel51   -693.1144580
## Channel52   1319.0783916
## Channel53  -1768.1923445
## Channel54   1559.5391421
## Channel55   -846.1912516
## Channel56    193.2670768
## Channel57    -17.8522107
## Channel58     -1.6218930
## Channel59    -95.7643127
## Channel60    365.5090328
## Channel61   -474.7783272
## Channel62    434.9185212
## Channel63   -160.3705721
## Channel64    197.7161715
## Channel65   -386.0347105
## Channel66    345.0906603
## Channel67   -333.0649717
## Channel68    209.6084918
## Channel69    -54.1848083
## Channel70   -308.4584111
## Channel71    286.4526990
## Channel72   -142.4377933
## Channel73    114.1458743
## Channel74    436.3227964
## Channel75    -19.1779733
## Channel76   -626.5032564
## Channel77    345.2654316
## Channel78    -80.0370682
## Channel79     -6.8450502
## Channel80   -687.5364098
## Channel81   1004.4289012
## Channel82   -873.2521467
## Channel83    417.6446929
## Channel84    374.4662265
## Channel85   -756.3190203
## Channel86    674.7705731
## Channel87    616.1113236
## Channel88  -2134.5659721
## Channel89   1680.1985410
## Channel90    307.9914475
## Channel91   -384.3839317
## Channel92   -408.6478848
## Channel93    189.3125294
## Channel94    259.5101250
## Channel95   -630.9519842
## Channel96    574.4308144
## Channel97   -141.5971464
## Channel98   -357.7614483
## Channel99    228.6726447
## Channel100    29.2001936
## Fat           -1.6496281
## Moisture      -0.9203932
\end{verbatim}

\begin{Shaded}
\begin{Highlighting}[]
\KeywordTok{cat}\NormalTok{(}\StringTok{"kappa:"}\NormalTok{, }\KeywordTok{kappa}\NormalTok{(A))}
\end{Highlighting}
\end{Shaded}

\begin{verbatim}
## kappa: 664318664630
\end{verbatim}

When repeating the steps with scaled data, R is able to solve the system
of equations. The kappa is also much smaller than previously. We
pressume that R can solve the equations now because the data is
normalized and all variables are now on the same scale.\\
Since we scaled the data, the values in matrix \(A\) are smaller than
the unscaled ones. This resulted in \(A\) being an approximately
well-conditioned matrix, shown by the new \(kappa(A)\) value which is
much smaller than before.

\newpage

\hypertarget{appendix}{%
\section{Appendix}\label{appendix}}

\begin{Shaded}
\begin{Highlighting}[]
\KeywordTok{RNGversion}\NormalTok{(}\KeywordTok{min}\NormalTok{(}\KeywordTok{as.character}\NormalTok{(}\KeywordTok{getRversion}\NormalTok{()), }\StringTok{"3.6.2"}\NormalTok{))}
\NormalTok{knitr}\OperatorTok{::}\NormalTok{opts_chunk}\OperatorTok{$}\KeywordTok{set}\NormalTok{(}\DataTypeTok{echo =} \OtherTok{TRUE}\NormalTok{)}
\KeywordTok{library}\NormalTok{(RMaCzek)}
\KeywordTok{library}\NormalTok{(knitr)}
\KeywordTok{library}\NormalTok{(tidyr)}
\KeywordTok{library}\NormalTok{(tidyverse)}
\KeywordTok{library}\NormalTok{(tinytex)}
\KeywordTok{library}\NormalTok{(dplyr)}
\KeywordTok{library}\NormalTok{(readxl)}
\NormalTok{x1 <-}\StringTok{ }\DecValTok{1}\OperatorTok{/}\DecValTok{3}\NormalTok{; x2 <-}\StringTok{ }\DecValTok{1}\OperatorTok{/}\DecValTok{4}
\ControlFlowTok{if}\NormalTok{(x1}\OperatorTok{-}\NormalTok{x2}\OperatorTok{==}\DecValTok{1}\OperatorTok{/}\DecValTok{12}\NormalTok{)\{}
  \KeywordTok{print}\NormalTok{(}\StringTok{"Subtraction is correct"}\NormalTok{)}
\NormalTok{\}}\ControlFlowTok{else}\NormalTok{\{}
  \KeywordTok{print}\NormalTok{(}\StringTok{"Subtraction is wrong"}\NormalTok{)}
\NormalTok{\}}
\NormalTok{x3 <-}\StringTok{ }\DecValTok{1}\NormalTok{; x4 <-}\StringTok{ }\DecValTok{1}\OperatorTok{/}\DecValTok{2}
\ControlFlowTok{if}\NormalTok{(x3}\OperatorTok{-}\NormalTok{x4}\OperatorTok{==}\DecValTok{1}\OperatorTok{/}\DecValTok{2}\NormalTok{)\{}
  \KeywordTok{print}\NormalTok{(}\StringTok{"Subtraction is correct"}\NormalTok{)}
\NormalTok{\}}\ControlFlowTok{else}\NormalTok{\{}
  \KeywordTok{print}\NormalTok{(}\StringTok{"Subtraction is wrong"}\NormalTok{)}
\NormalTok{\}}
\KeywordTok{sprintf}\NormalTok{(}\StringTok{"%.20f"}\NormalTok{, (x1}\OperatorTok{-}\NormalTok{x2))}
\KeywordTok{sprintf}\NormalTok{(}\StringTok{"%.20f"}\NormalTok{, }\DecValTok{1}\OperatorTok{/}\DecValTok{12}\NormalTok{)}
\NormalTok{x1 <-}\StringTok{ }\DecValTok{1}\OperatorTok{/}\DecValTok{3}\NormalTok{; x2 <-}\StringTok{ }\DecValTok{1}\OperatorTok{/}\DecValTok{4}
\ControlFlowTok{if}\NormalTok{(}\KeywordTok{round}\NormalTok{((x1}\OperatorTok{-}\NormalTok{x2), }\DataTypeTok{digits =} \DecValTok{15}\NormalTok{) }\OperatorTok{==}\StringTok{ }\KeywordTok{round}\NormalTok{((}\DecValTok{1}\OperatorTok{/}\DecValTok{12}\NormalTok{), }\DataTypeTok{digits =} \DecValTok{15}\NormalTok{))\{}
  \KeywordTok{print}\NormalTok{(}\StringTok{"Subtraction is correct"}\NormalTok{)}
\NormalTok{\}}\ControlFlowTok{else}\NormalTok{\{}
  \KeywordTok{print}\NormalTok{(}\StringTok{"Subtraction is wrong"}\NormalTok{)}
\NormalTok{\}}
\NormalTok{f <-}\StringTok{ }\ControlFlowTok{function}\NormalTok{(x)\{}
  \KeywordTok{return}\NormalTok{(x)}
\NormalTok{\}}

\NormalTok{myderiv <-}\StringTok{ }\ControlFlowTok{function}\NormalTok{(f, x, e)\{}
\NormalTok{  deriv <-}\StringTok{ }\NormalTok{(}\KeywordTok{f}\NormalTok{(x }\OperatorTok{+}\StringTok{ }\NormalTok{e) }\OperatorTok{-}\StringTok{ }\KeywordTok{f}\NormalTok{(x))}\OperatorTok{/}\NormalTok{(e)}
  \KeywordTok{return}\NormalTok{(deriv)}
\NormalTok{\}}
\KeywordTok{myderiv}\NormalTok{(}\DataTypeTok{f =}\NormalTok{ f, }\DataTypeTok{x =} \DecValTok{1}\NormalTok{, }\DataTypeTok{e =} \DecValTok{10}\OperatorTok{^-}\DecValTok{15}\NormalTok{)}
\KeywordTok{myderiv}\NormalTok{(}\DataTypeTok{f =}\NormalTok{ f, }\DataTypeTok{x =} \DecValTok{100000}\NormalTok{, }\DataTypeTok{e =} \DecValTok{10}\OperatorTok{^-}\DecValTok{15}\NormalTok{)}
\NormalTok{myvar <-}\StringTok{ }\ControlFlowTok{function}\NormalTok{(x)\{}
\NormalTok{  n <-}\StringTok{ }\KeywordTok{length}\NormalTok{(x)}
\NormalTok{  variance <-}\StringTok{ }\NormalTok{(}\DecValTok{1}\OperatorTok{/}\NormalTok{(n}\DecValTok{-1}\NormalTok{)) }\OperatorTok{*}\StringTok{ }\NormalTok{(}\KeywordTok{sum}\NormalTok{(x}\OperatorTok{^}\DecValTok{2}\NormalTok{) }\OperatorTok{-}\StringTok{ }\NormalTok{((}\DecValTok{1}\OperatorTok{/}\NormalTok{n)}\OperatorTok{*}\NormalTok{(}\KeywordTok{sum}\NormalTok{(x))}\OperatorTok{^}\DecValTok{2}\NormalTok{))}
  \KeywordTok{return}\NormalTok{(variance)}
\NormalTok{\}}
\KeywordTok{set.seed}\NormalTok{(}\DecValTok{12345}\NormalTok{, }\DataTypeTok{kind =} \StringTok{"Mersenne-Twister"}\NormalTok{, }\DataTypeTok{normal.kind =} \StringTok{"Inversion"}\NormalTok{)}
\NormalTok{data <-}\StringTok{ }\KeywordTok{rnorm}\NormalTok{(}\DecValTok{10000}\NormalTok{, }\DataTypeTok{mean=}\DecValTok{10}\OperatorTok{^}\DecValTok{8}\NormalTok{, }\DataTypeTok{sd=}\DecValTok{1}\NormalTok{)}
\NormalTok{Yi <-}\StringTok{ }\DecValTok{0}

\ControlFlowTok{for}\NormalTok{(i }\ControlFlowTok{in} \DecValTok{1}\OperatorTok{:}\KeywordTok{length}\NormalTok{(data))\{}
\NormalTok{  Yi[i] <-}\StringTok{ }\KeywordTok{myvar}\NormalTok{(data[}\DecValTok{1}\OperatorTok{:}\NormalTok{i]) }\OperatorTok{-}\StringTok{ }\KeywordTok{var}\NormalTok{(data[}\DecValTok{1}\OperatorTok{:}\NormalTok{i])}
\NormalTok{\}}

\NormalTok{i <-}\StringTok{ }\KeywordTok{c}\NormalTok{(}\DecValTok{1}\OperatorTok{:}\DecValTok{10000}\NormalTok{)}
  
\KeywordTok{plot}\NormalTok{(i, Yi)}
\NormalTok{newvar <-}\StringTok{ }\ControlFlowTok{function}\NormalTok{(x)\{}
\NormalTok{  n <-}\StringTok{ }\KeywordTok{length}\NormalTok{(x)}
\NormalTok{  xbar <-}\StringTok{ }\KeywordTok{rep}\NormalTok{(}\KeywordTok{mean}\NormalTok{(x), n)}
\NormalTok{  variance <-}\StringTok{ }\KeywordTok{sum}\NormalTok{(((x }\OperatorTok{-}\StringTok{ }\NormalTok{xbar)}\OperatorTok{^}\DecValTok{2}\NormalTok{))}\OperatorTok{/}\NormalTok{(n}\DecValTok{-1}\NormalTok{)}
  \KeywordTok{return}\NormalTok{(variance)}
\NormalTok{\}}

\NormalTok{Yi2 <-}\StringTok{ }\DecValTok{0}

\ControlFlowTok{for}\NormalTok{(i }\ControlFlowTok{in} \DecValTok{1}\OperatorTok{:}\KeywordTok{length}\NormalTok{(data))\{}
\NormalTok{  Yi2[i] <-}\StringTok{ }\KeywordTok{newvar}\NormalTok{(data[}\DecValTok{1}\OperatorTok{:}\NormalTok{i]) }\OperatorTok{-}\StringTok{ }\KeywordTok{var}\NormalTok{(data[}\DecValTok{1}\OperatorTok{:}\NormalTok{i])}
\NormalTok{\}}

\NormalTok{i <-}\StringTok{ }\KeywordTok{c}\NormalTok{(}\DecValTok{1}\OperatorTok{:}\DecValTok{10000}\NormalTok{)}

\KeywordTok{plot}\NormalTok{(i, Yi2)}

\NormalTok{data <-}\StringTok{ }\KeywordTok{read_excel}\NormalTok{(}\StringTok{"tecator.xls"}\NormalTok{)}
\NormalTok{y_index <-}\StringTok{ }\KeywordTok{grep}\NormalTok{(}\StringTok{"Protein"}\NormalTok{, }\KeywordTok{colnames}\NormalTok{(data))}
\NormalTok{y <-}\StringTok{ }\KeywordTok{as.matrix}\NormalTok{(data}\OperatorTok{$}\NormalTok{Protein)}
\NormalTok{X <-}\StringTok{ }\KeywordTok{as.matrix}\NormalTok{(data[,}\OperatorTok{-}\NormalTok{y_index])}
\NormalTok{A <-}\StringTok{ }\KeywordTok{t}\NormalTok{(X) }\OperatorTok\StringTok{ }\NormalTok{X}
\NormalTok{b <-}\StringTok{  }\KeywordTok{t}\NormalTok{(X) }\OperatorTok\StringTok{ }\NormalTok{y  }
\KeywordTok{solve}\NormalTok{(A, b)}
\KeywordTok{kappa}\NormalTok{(A)}
\NormalTok{data2 <-}\StringTok{ }\KeywordTok{scale}\NormalTok{(data)}
\NormalTok{y_index <-}\StringTok{ }\KeywordTok{grep}\NormalTok{(}\StringTok{"Protein"}\NormalTok{, }\KeywordTok{colnames}\NormalTok{(data2))}
\NormalTok{y <-}\StringTok{ }\KeywordTok{as.matrix}\NormalTok{(data2[,y_index])}
\NormalTok{X <-}\StringTok{ }\KeywordTok{as.matrix}\NormalTok{(data2[,}\OperatorTok{-}\NormalTok{y_index])}
\NormalTok{A <-}\StringTok{ }\KeywordTok{t}\NormalTok{(X) }\OperatorTok\StringTok{ }\NormalTok{X}
\NormalTok{b <-}\StringTok{  }\KeywordTok{t}\NormalTok{(X) }\OperatorTok\StringTok{ }\NormalTok{y  }
\KeywordTok{solve}\NormalTok{(A, b)}
\KeywordTok{cat}\NormalTok{(}\StringTok{"kappa:"}\NormalTok{, }\KeywordTok{kappa}\NormalTok{(A))}
\end{Highlighting}
\end{Shaded}


\end{document}
